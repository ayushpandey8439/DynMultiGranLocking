\documentclass{paper}

\usepackage{fullpage}
\usepackage{amsmath}
\usepackage{amssymb}
\usepackage{theorem}
\usepackage{xcolor}
\usepackage{todonotes}

\newtheorem{theorem}{Theorem}
\newtheorem{definition}{Definition}
\newtheorem{lemma}{Lemma}
\newtheorem{corollary}{Corollary}

\newenvironment{proof}{{\noindent\bf Proof. } }{{\hfill $\Box$}}

\begin{document}

\section{Definitions}

Let $G=(V,E)$ be a rooted DAG of $n$ nodes. We denote its roots by $R$ and its set of (directed) paths $\mathcal{P}$. For a given path $p\in\mathcal{P}$, we denote its source $s(p)$, its destination $d(p)$ and its length $l(p)$.

Given a node $v$, its depht $\delta(v)$ is its distance to the root $r$.

Given a node $v$, its succesors are the nodes of the set: $S_v=\{u\in V|\exists e\in E, e=\{v,u\}\}$.

Given a node $v$, its predecessors are the nodes of the set: $P_v=\{u\in V|\exists e\in E, e=\{u,v\}\}$.

Given a node $v$, its ancestors are the nodes of the set: $A_v=\{u\in V|\exists p\in \mathcal{P}, s(p)=v\wedge d(p)=u\}$.

Given a node $v$, its descendents are the nodes of the set: $D_v=\{u\in V|\exists p\in \mathcal{P},  s(p)=u \wedge d(p)=v\}$.

\subsection{Critical ancestor of a node}

Given a node $v$, its critical ancestors are the nodes of the set:
\[
CA_v=\{u\in A_v|\forall p\in \mathcal{P}, s(p)\in R\wedge d(p)=v\Rightarrow u\in p\}
\]


\textcolor{red}{Given a node $v$, its lowest critical ancestor (denoted $LCA_v$) is the node $u$ of $CA_v$ satisfying $D_u\cap CA_v=\emptyset$. The lowest critical ancestor is always unique.} \todo{we dont need this definition.}

\subsection{Common critical ancestors of a pair of nodes}
Given two nodes $u$ and $v$, their common critical ancestors are the nodes of the set:
\[
CCA_{u,v} = \{w\in A_u \cap A_v |\forall p\in \mathcal{P}, s(p)\in R\wedge (d(p)=u \lor d(p) = v)\Rightarrow w\in p\}
\]

This definition can also be summarised as:

\[
CCA_{u,v} = \{w\in V | w \in CA_u \wedge w \in CA_v\}
\]

\subsection{Lowest common critical ancestor of a pair of nodes}

Given two nodes $u$ and $v$, their lowest common critical ancestor (denoted $LCCA_{u,v}$) is the node $w$ of $CSA_{u,v}$ satisfying $D_w\cap CA_{u,v}=\emptyset$.


\subsection{Lowest common single ancestor of two nodes}

\todo{A few meetings back we decided to give up the term single ancestor. This is because computing the lowest ancestor inherently gives us a single node and calling ancestors common and single was confusing. So we choose the name "lowest common critical ancestor". }

Given two nodes $u$ and $v$, their common ancestors are the nodes of the set:

\[
CA_{u,v}=\{w\in V|(\exists p\in\mathcal{P},s(p)=w\wedge d(p)=v)\wedge(\exists p\in\mathcal{P},s(p)=w\wedge d(p)=u)\}
\].

\textcolor{red}{equivalent to $CA_{u,v}=A_v\cap A_u$} \textcolor{orange}{No}

Given two nodes $u$ and $v$, their common single ancestors are the nodes of the set: $CSA_{u,v}=\{w\in CA_{u,v}|(\forall p\in\mathcal{P},s(p)=r\wedge d(p)=u\Rightarrow w\in p)\wedge(\forall p\in\mathcal{P},s(p)=r\wedge d(p)=v\Rightarrow w\in p) \}$.

Given two nodes $u$ and $v$, their lowest common single ancestors (denoted $LCSA_{u,v}$) is the node $w$ of $CSA_{u,v}$ satisfying $D_w\cap CSA_{u,v}=\emptyset$.

\section{Main Property}
{\color{red}
What is the property between $LCSA_{u,v}$, $CA_u$, and $CA_v$ used by our algorithm?

\color{orange}
Our algorithm defines for every node the set $CA_u$. We can then use this set to compute the $CCA_{u,v}$ by taking the set intersection as defined earlier.


\color{red}

If we can find a relation with the depth, the proof will become easy! Is the following property correct?

Given two nodes $u$ and $v$, $LCSA_{u,v}$ is the node of $CA_u\cap CA_v$ with the highest depth.

\color{orange}
This is infact correct and except the name. $LCSA_{u,v}$ is $LCCA_{u,v}$
\color{red}
We have to prove this property!

}

%\begin{theorem}
%For any two nodes $u$ and $v$ of $V$, $LCSA_{u,v}=LCSA_{LCA_u,LCA_v}$.
%\end{theorem}

%\begin{proof}
%TO DO
%\end{proof}

\section{Labeling Algorithm}

\subsection{Algorithm}

Given a rooted DAG $G$, we compute (with a breadth-first traversal) for each node $v$ a set $S_v$ according to the following rule:
\[
S_v=\begin{cases}
\{v\} \text{ if } v=r\\
\{v\}\cup\{\cap_{u\in P_v} S_u\} \text{ otherwise}
\end{cases}
\]

\textcolor{red}{Association of each node with its depht (for the LCSA computation)?}

%Once done, the Labeling Algorithm returns, for each node $v$, the node $r_v$ of $S_v$ which has the highest depth (also computed during the breadth-first traversal).

\subsection{Proof}

\begin{lemma}
For each node $v\in V$, the set $S_v$ computed by the Labeling Algorithm is $CA_v$.
\end{lemma}

\begin{proof}
\textcolor{red}{TO DO}
\end{proof}

\section{LCSA Algorithm}

\subsection{Algorithm}

{\color{red}What does exactly this algorithm?

I believe that it is something like that (but I am not really sure):

Given a rooted DAG $G$ labeled by the Labeling Algorithm, and two nodes $u,v\in V$, the LCSA Algorithm returns the node of $S_u\cap S_v$ with the highest depth.
}

\subsection{Proof}

\begin{lemma}
For each pair of node $u,v\in V$, the node computed by the LCSA Algorithm is $LCSA_{u,v}$.
\end{lemma}

\begin{proof}
\textcolor{red}{TO DO - easy if the main property is true!}
\end{proof}

\end{document}
